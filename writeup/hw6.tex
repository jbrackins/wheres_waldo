%%%%%%%%%%%%%%%%%%%%%%%%%%%%%%%%%%%%%%%%%
% Programming/Coding Assignment
% LaTeX Template
%
% This template has been downloaded from:
% http://www.latextemplates.com
%
% Original author:
% Ted Pavlic (http://www.tedpavlic.com)
%
% Note:
% The \lipsum[#] commands throughout this template generate dummy text
% to fill the template out. These commands should all be removed when 
% writing assignment content.
%
% This template uses a Perl script as an example snippet of code, most other
% languages are also usable. Configure them in the "CODE INCLUSION 
% CONFIGURATION" section.
%
%%%%%%%%%%%%%%%%%%%%%%%%%%%%%%%%%%%%%%%%%

%----------------------------------------------------------------------------------------
%	PACKAGES AND OTHER DOCUMENT CONFIGURATIONS
%----------------------------------------------------------------------------------------

\documentclass{article}

\usepackage{fancyhdr} % Required for custom headers
\usepackage{lastpage} % Required to determine the last page for the footer
\usepackage{extramarks} % Required for headers and footers
\usepackage[usenames,dvipsnames]{color} % Required for custom colors
\usepackage{graphicx} % Required to insert images
\usepackage{listings} % Required for insertion of code
\usepackage{courier} % Required for the courier font
\usepackage{lipsum} % Used for inserting dummy 'Lorem ipsum' text into the template

\usepackage{enumerate}
\usepackage{amsmath}
\usepackage{hyperref}

\usepackage{listings}
\usepackage{color} %red, green, blue, yellow, cyan, magenta, black, white
\definecolor{mygreen}{RGB}{28,172,0} % color values Red, Green, Blue
\definecolor{mylilas}{RGB}{170,55,241}

% Margins
\topmargin=-0.45in
\evensidemargin=0in
\oddsidemargin=0in
\textwidth=6.5in
\textheight=9.0in
\headsep=0.25in

\linespread{1.1} % Line spacing

% Set up the header and footer
\pagestyle{fancy}
\lhead{\hmwkAuthorName} % Top left header
\chead{\hmwkClass\ (\hmwkClassInstructor\ \hmwkClassTime): \hmwkTitle} % Top center head
\rhead{\firstxmark} % Top right header
\lfoot{\lastxmark} % Bottom left footer
\cfoot{} % Bottom center footer
\rfoot{Page\ \thepage\ of\ \protect\pageref{LastPage}} % Bottom right footer
\renewcommand\headrulewidth{0.4pt} % Size of the header rule
\renewcommand\footrulewidth{0.4pt} % Size of the footer rule

\setlength\parindent{0pt} % Removes all indentation from paragraphs

%----------------------------------------------------------------------------------------
%	CODE INCLUSION CONFIGURATION
%----------------------------------------------------------------------------------------

\definecolor{MyDarkGreen}{rgb}{0.0,0.4,0.0} % This is the color used for comments
\lstloadlanguages{Perl} % Load Perl syntax for listings, for a list of other languages supported see: ftp://ftp.tex.ac.uk/tex-archive/macros/latex/contrib/listings/listings.pdf
\lstset{language=Perl, % Use Perl in this example
        frame=single, % Single frame around code
        basicstyle=\small\ttfamily, % Use small true type font
        keywordstyle=[1]\color{Blue}\bf, % Perl functions bold and blue
        keywordstyle=[2]\color{Purple}, % Perl function arguments purple
        keywordstyle=[3]\color{Blue}\underbar, % Custom functions underlined and blue
        identifierstyle=, % Nothing special about identifiers                                         
        commentstyle=\usefont{T1}{pcr}{m}{sl}\color{MyDarkGreen}\small, % Comments small dark green courier font
        stringstyle=\color{Purple}, % Strings are purple
        showstringspaces=false, % Don't put marks in string spaces
        tabsize=5, % 5 spaces per tab
        %
        % Put standard Perl functions not included in the default language here
        morekeywords={rand},
        %
        % Put Perl function parameters here
        morekeywords=[2]{on, off, interp},
        %
        % Put user defined functions here
        morekeywords=[3]{test},
       	%
        morecomment=[l][\color{Blue}]{...}, % Line continuation (...) like blue comment
        numbers=left, % Line numbers on left
        firstnumber=1, % Line numbers start with line 1
        numberstyle=\tiny\color{Blue}, % Line numbers are blue and small
        stepnumber=5 % Line numbers go in steps of 5
}

% Creates a new command to include a perl script, the first parameter is the filename of the script (without .pl), the second parameter is the caption
\newcommand{\perlscript}[2]{
\begin{itemize}
\item[]\lstinputlisting[caption=#2,label=#1]{#1.pl}
\end{itemize}
}

%----------------------------------------------------------------------------------------
%	DOCUMENT STRUCTURE COMMANDS
%	Skip this unless you know what you're doing
%----------------------------------------------------------------------------------------

% Header and footer for when a page split occurs within a problem environment
\newcommand{\enterProblemHeader}[1]{
\nobreak\extramarks{#1}{#1 continued on next page\ldots}\nobreak
\nobreak\extramarks{#1 (continued)}{#1 continued on next page\ldots}\nobreak
}

% Header and footer for when a page split occurs between problem environments
\newcommand{\exitProblemHeader}[1]{
\nobreak\extramarks{#1 (continued)}{#1 continued on next page\ldots}\nobreak
\nobreak\extramarks{#1}{}\nobreak
}

\setcounter{secnumdepth}{0} % Removes default section numbers
\newcounter{homeworkProblemCounter} % Creates a counter to keep track of the number of problems

\newcommand{\homeworkProblemName}{}
\newenvironment{homeworkProblem}[1][Problem \arabic{homeworkProblemCounter}]{ % Makes a new environment called homeworkProblem which takes 1 argument (custom name) but the default is "Problem #"
\stepcounter{homeworkProblemCounter} % Increase counter for number of problems
\renewcommand{\homeworkProblemName}{#1} % Assign \homeworkProblemName the name of the problem
\section{\homeworkProblemName} % Make a section in the document with the custom problem count
\enterProblemHeader{\homeworkProblemName} % Header and footer within the environment
}{
\exitProblemHeader{\homeworkProblemName} % Header and footer after the environment
}

\newcommand{\problemAnswer}[1]{ % Defines the problem answer command with the content as the only argument
\noindent\framebox[\columnwidth][c]{\begin{minipage}{0.98\columnwidth}#1\end{minipage}} % Makes the box around the problem answer and puts the content inside
}

\newcommand{\homeworkSectionName}{}
\newenvironment{homeworkSection}[1]{ % New environment for sections within homework problems, takes 1 argument - the name of the section
\renewcommand{\homeworkSectionName}{#1} % Assign \homeworkSectionName to the name of the section from the environment argument
\subsection{\homeworkSectionName} % Make a subsection with the custom name of the subsection
\enterProblemHeader{\homeworkProblemName\ [\homeworkSectionName]} % Header and footer within the environment
}{
\enterProblemHeader{\homeworkProblemName} % Header and footer after the environment
}

%----------------------------------------------------------------------------------------
%	NAME AND CLASS SECTION
%----------------------------------------------------------------------------------------

\newcommand{\hmwkTitle}{Homework\ \#6} % Assignment title
\newcommand{\hmwkDueDate}{Thursday,\ October\ 22,\ 2015} % Due date
\newcommand{\hmwkClass}{CSC 514: Computer Vision} % Course/class
\newcommand{\hmwkClassTime}{11:00am} % Class/lecture time
\newcommand{\hmwkClassInstructor}{Hoover} % Teacher/lecturer
\newcommand{\hmwkAuthorName}{Julian A. Brackins} % Your name

%----------------------------------------------------------------------------------------
%	TITLE PAGE
%----------------------------------------------------------------------------------------

\title{
\vspace{2in}
\textmd{\textbf{\hmwkClass:\ \hmwkTitle}}\\
\normalsize\vspace{0.1in}\small{Due\ on\ \hmwkDueDate}\\
\vspace{0.1in}\large{\textit{\hmwkClassInstructor\ \hmwkClassTime}}
\vspace{3in}
}

\author{\textbf{\hmwkAuthorName}}
\date{} % Insert date here if you want it to appear below your name

%----------------------------------------------------------------------------------------

\begin{document}

\maketitle

%----------------------------------------------------------------------------------------
%	TABLE OF CONTENTS
%----------------------------------------------------------------------------------------

%\setcounter{tocdepth}{1} % Uncomment this line if you don't want subsections listed in the ToC

\newpage

\newpage

%----------------------------------------------------------------------------------------
%	PROBLEM 1
%----------------------------------------------------------------------------------------

% To have just one problem per page, simply put a \clearpage after each problem

\begin{homeworkProblem}
In class we discussed several different methods for performing filtering/template matching on images. In this homework you will create your own Matlab functions to perform both convolution and correlation on images using different convolution/correlation kernels.\\ \\
\noindent


\lstset{language=Matlab,%
    %basicstyle=\color{red},
    breaklines=true,%
    morekeywords={matlab2tikz},
    keywordstyle=\color{blue},%
    morekeywords=[2]{1}, keywordstyle=[2]{\color{black}},
    identifierstyle=\color{black},%
    stringstyle=\color{mylilas},
    commentstyle=\color{mygreen},%
    showstringspaces=false,%without this there will be a symbol in the places where there is a space
    numbers=left,%
    numberstyle={\tiny \color{black}},% size of the numbers
    numbersep=9pt, % this defines how far the numbers are from the text
    emph=[1]{for,end,break},emphstyle=[1]\color{red}, %some words to emphasise
    %emph=[2]{word1,word2}, emphstyle=[2]{style},    
}

\clearpage
\begin{lstlisting}[caption = {Where's Waldo Script}]
function wheres_waldo()
    %%Read in Kernel, convert to grayscale double. Save original copy.
    kernel = imread('WaldoKernel.png');
    orig_kern = kernel;
    kernel = rgb2gray(kernel);
    kernel = im2double(kernel);

    %%Read in Scene, convert to grayscale double. Save original copy.
    scene = imread('WaldoScene.png');
    orig_scene = scene;
    scene = rgb2gray(scene);
    scene = im2double(scene);

    %%Find the dimensions for the scene and kernel
    [kH, kW] = size(kernel);
    [sH, sW] = size(scene);
    
     %%Divide Kernel Height and Width
    kH = kH/2;
    kW = kW/2;
    
    %%Calculate Correlation
    G =  correlation( scene, kernel );
    %%OR you can do convolution...
    %%G =  convolution( scene, kernel );
    %%Actually don't because convlution is trash for this program...

    %%Find Waldo by finding the highest value point
    [r,c] = find(G==max(G(:)));

    %%Pad array so that the imposed image lines up properly
    scene  = padarray(scene,[kH,kW]);
    orig_kern = padarray(orig_kern,[r,c],'pre');
    orig_kern = padarray(orig_kern,[sH-r,sW-c],'post');

    %%Show the Original Image
    figure, imshow(orig_scene,[]);
    title('Original Scene');

    %%Show the surf Image
    figure, surf(G), shading flat;
    title('Quantized Samples');

    %%Show the G matrix
    figure, imshow(G,[]);
    title('Waldo Guess location');

    %%Show the Search Result
    figure, imshowpair(scene(:,:,1),orig_kern,'blend');
    title('Kernel Superimposed on Original Image');

end
\end{lstlisting}

\clearpage


\begin{lstlisting}[caption = {Correlation Function}]
function [ G ] = correlation( scene, kernel )
    %%Find the dimensions for the scene and kernel
    [kH, kW] = size(kernel);
    [sH, sW] = size(scene);


    %%set F, G, H matrices so that they match what's in the book.
    G = scene;
    F = scene;
    H = kernel;

    %%Divide Kernel Height and Width so we work with a smaller kern
    kH = kH/2;
    kW = kW/2;

    %%Generate Mean for Scene and Kernel
    meanS = mean(mean(scene));
    meanK = mean(mean(kernel));


    %%Perform Correlation
    for i=(kH):(sH-kH)
        for j=(kW):(sW-kW)
            G(i,j) = sum(sum((F(i-kH+1:i+kH, j-kW+1:j+kW)-meanS).*(H-meanK)));
        end
    end
end
\end{lstlisting}

\begin{lstlisting}[caption = {Convolution Function}]
    %%For Convolution, just Flip the filter in both directions
    kernel = rot90(kernel,2);
    [kH, kW] = size(kernel);
    [sH, sW] = size(scene);

    G = scene;
    F = scene;
    H = kernel;

    kH = kH/2;
    kW = kW/2;

    meanS = mean(mean(scene));
    meanK = mean(mean(kernel));

    %%Perform Correlation
    for i=(kH):(sH-kH)
        for j=(kW):(sW-kW)
            G(i,j) = sum(sum((F(i-kH+1:i+kH, j-kW+1:j+kW)-meanS).*(H-meanK)));
        end
    end
end
\end{lstlisting}

\end{homeworkProblem}
\clearpage

%----------------------------------------------------------------------------------------

\end{document}